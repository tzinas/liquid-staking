\pdfbookmark[section]{Representation}{Representation}
\section{Representation}

In a staking protocol, whether liquid or not, each principal has an opinion
about which validator they wish to delegate to.

\begin{definition}[Delegation Wish]
  For each principal, we define
  their \emph{delegation wish} to be a particular validator of their choice.
\end{definition}

It will generally be desired that these delegation wishes point to
honest validators. We formalize this in the notion of \emph{wisdom}.

\begin{definition}[Wisdom]
  A principal is \emph{wise} if their delegation wish points to an honest
  validator. Otherwise, the principal is \emph{unwise}.
\end{definition}

In the liquid staking protocols described in Section~\ref{sec:preliminaries},
the decision of which validators to delegate to is up to the majority of the
stakeholders.
%TODO (w/ dionyziz): We also talked about centralized protocols, should we specify?
This creates a problem. A stakeholder holding a minority of the stake
may wish to delegate this stake to a particular validator, but the rest
of the stakeholders can overturn him by a majority vote. Hence, in these
protocols we have a situation of \emph{only the majority} being represented,
instead of \emph{everyone being equally represented}~\cite{mill1862true}.

On the contrary, in a \emph{proportional representation} system, the
majority of the stakeholders decide where to delegate the majority of
the stake, but the minority of the stakeholders also decide where to delegate
the minority of the stake.

% TODO (e-print): Formalize proportional representation.

% Formal version:
% TODO(dionyziz): Consider moving this to the appendix.
% \[
%   \mathcal{D}, \mathcal{V}
%
%   v: \mathcal{D} \longrightarrow \mathbb{R}
%
%   f: \mathcal{D} \longrightarrow \mathcal{V}
%
%   \text{Proportional representation:}
%
%   w: \mathcal{V} \longrightarrow \mathbb{R}
%   \sum_{v \in \mathcal{V}} w(v) = \sum_{d \in \mathcal{D}} v(d)
%
%   \forall v \in \mathcal{V}:
%   w(v) = \sum_{d \in \mathcal{D}\\f(d) = v} v(d)
% \]
\begin{definition}[Proportional Representation]
  In a \emph{proportionally represented} liquid staking protocol,
  each validator is delegated a proportion of the liquid staking pool's \asset
  equal to the sum of the proportions of \stasset held by the principals
  who wish to delegate to that validator.
\end{definition}
% TODO (w/ dionyziz): Can this be simplified?

To achieve this, the process of liquid
staking becomes different: Each principal must \emph{signal} their
\emph{intent} indicating which validator they wish
the pool to delegate to. This election mechanism was introduced by
QuickSilver~\cite{quicksilver} even though it is not yet implemented at
the time of writing.

First, the principal deposits \asset into the protocol and signals
delegation intent to the validator of their choice. Then, the protocol
delegates the deposited \asset to that validator. The principal is
now a holder of tradable \stasset representations of the delegated \asset.
This \stasset can now be transferred to a different owner.

At some later time, the \stasset holder may wish to redelegate
their underlying \asset to a new validator.
The \stasset holder can \emph{resignal} their delegation intent and the protocol
will redelegate the underlying \asset to the new validator.
The goal is for all \stasset holders to be represented proportionally to
their stake.

Note that proportional representation may not be instant. Redelegation speeds
are limited by the underlying blockchains' unbonding period.
Hence, a \stasset holder may have to wait $\delta$ before their corresponding
\assets are redelegated to the validator of their choice.