\section{Representation}
% TODO: introduce refungibilization

% Stake -> Delegated Stake -> Tokenized Stake -> Refungibilized Stake
% .    \
% .     -> Exempt Delegation
In the liquid staking protocols described in Section~\ref{sec:preliminaries},
the decision of which validators to delegate to is up to the governance.
This creates a problem. A stakeholder holding a minority of the stake
may wish to delegate this stake to a particular validator, but the rest
of the stakeholders can overturn him by a majority vote. Hence, in these
protocols we have a situation of \emph{only the majority} being represented,
instead of \emph{everyone being equally represented}~\cite{mill1862true}.

On the contrary, in a \emph{proportional representation} system, the
majority of the stakeholders decide where to delegate the majority of
the stake, but the minority of the stakeholders also decide where to delegate
the minority of the stake. In order to achieve this, the process of liquid
staking becomes different. Each stakeholder must \emph{signal} their
\emph{intent}~\cite{quicksilver} indicating which validator they wish
the pool to delegate to.

First, the stakeholder delegates their stake
directly to the validator of their choice. They are now holders of \emph{delegated
stake}, which is non-fungible and illiquid. They then
transfer\footnote{In some Cosmos implementations, illiquid delegated stake requires an intermediate
step before it can transferred: making the delegated stake \emph{minimally fungible}.
This is done by first converting the delegated stake into minimally fungible
\emph{liquid staking tokens} using the Liquidity Staking Module (LSM)~\cite{liquidity-staking-module}.
These tokens are then transfered to the liquid staking protocol for refungibilization.}
this delegated stake to the liquid staking protocol, which issues derivative
tokens to the stakeholder. These tokens are fully fungible. We call this
process \emph{refungibilization}, because the user starts with fungible assets,
which we denote \textsf{Asset}, converts them to non-fungible delegated stake,
which we denote \textsf{dAsset}, and, finally, deposites these into the liquid
staking protocol to receive a fully fungible derivative token, which we denote
\textsf{stAsset}.

Asset, stAsset

This means that, if a small stakeholder
participating in the liquid staking protocol wants their... TODO

It makes sense to wrap this delegated
stake into a \emph{derivative token} which can be used for trading and can
be swapped for the underlying delegated stake at any time. Such tokens
are most useful if they are \emph{fungible} with one another, regardless
of the validator they originated from. We use the term
\emph{refungibilization} for the process of converting (non-fungible) delegated stake
into fungible tokens.

\begin{itemize}
    \item Signalling on entering, resignalling during exchange
\end{itemize}
