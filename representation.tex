\section{Representation}

In the liquid staking protocols described in Section~\ref{sec:preliminaries},
the decision of which validators to delegate to is up to the governance.
This creates a problem. A stakeholder holding a minority of the stake
may wish to delegate this stake to a particular validator, but the rest
of the stakeholders can overturn him by a majority vote. Hence, in these
protocols we have a situation of \emph{only the majority} being represented,
instead of \emph{everyone being equally represented}~\cite{mill1862true}.

On the contrary, in a \emph{proportional representation} system, the
majority of the stakeholders decide where to delegate the majority of
the stake, but the minority of the stakeholders also decide where to delegate
the minority of the stake. In order to achieve this, the process of liquid
staking becomes different. Each stakeholder must \emph{signal} their
\emph{intent}~\cite{quicksilver} indicating which validator they wish
the pool to delegate to.

First, the stakeholder deposits \asset into the protocol and signals
delegation intent to the validator of their choice. Then, the protocol
delegates the deposited \asset to that validator. The stakeholder is
now a holder of tradable \stasset representations of the delegated \asset.

At some later time, the \stasset holder may wish to redelegate
their underlying \asset to a new validator.
The \stasset holder can \emph{resignal} their delegation intent and the protocol
will redelegate the underlying \asset to the new validator.

The goal is for all \stasset holders to be represented proportionally to
their stake. Hence, in a liquid staking protocol with proportional
representation, for every \stasset in circulation, the underlying \asset
in the protocol's delegation pool must be delegated to the validator
of the current \stasset holder's choice.

Note that proportional representation may not be instant. Redelegation speeds
are limited by the underlying blockchains' unbonding period.
Hence, a \stasset holder may have to wait $delta$ before their corresponding
\assets are redelegated to the validator of their choice. This is called
eventual proportion representation.