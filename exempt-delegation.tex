\section{Exempt Delegations}

Exempt delegations (proposed in LSM~\cite{liquidity-staking-module})
are a mechanism to alleviate
the Principal--Agent problem in liquid staking.
Delegators can choose to make an \emph{exempt delegation},
instead of a regular delegation, to a validator.
These exempt delegations, as their name suggests,
are \emph{exempt} from tokenization,
and are akin to self-delegation of older systems.
In case of validator misbehavior, exempt delegations
are slashed at a larger proportion $0 < q \leq 1$ than regular delegations.
Therefore, exempt delegations act as a meter of the
validator's trustworthiness.

% TODO: $\phi$
% Let $\varphi$ be the exempt delegation factor of the protocol
% that determines the amount $b$ as follows: $b \leq \varphi c$.

The amount of \asset in the protocol's delegation pool that can be
delegated to a specific validator is limited by the amount of its
exempt delegations and the protocol's \emph{Exempt Delegation Factor} $\varphi$.
Hence, if validator $\mathcal{V}$ has $c$ \asset worth of exempt delegations,
the amount of \asset that can be delegated to $\mathcal{V}$ by the protocol
is $b \leq \varphi c$.

The attack described in the previous section can be modified to work
around exempt delegations.

Like before, the adversary obtains capital $b$ \asset and creates
validator $\mathcal{V}$.
However, $\mathcal{V}$ has no exempt delegations yet and cannot be delegated to
by the protocol.
Therefore, she additionally obtains capital $c = \frac{b}{\varphi}$ \asset
and at time $t_0 < t_1$ uses it to exempt delegate to $\mathcal{V}$.
Now, $\mathcal{V}$ has enough exempt delegations to be delegated to by the protocol.

Then, once again, the adversary delegates $b$ \asset at time $t_1$,
sells the acquired $s$ \stasset in the market at time $t_2$, shorts \stasset by taking a loan
of $z$ \stasset at time $t_3$, equivocates using $\mathcal{V}$ at time $t_4$.

% TODO: Clean up the extension of the attack using the c parameter below

The adversary starts off by creating a validator and making an
exempt delegating of $c$ ATOM to it.
Then make a regular delegation of $b$ ATOM to the validator and
immediately convert all the regular delegation shares to $b$ liquid
staking tokens.

Then, the adversary converts all the $b$ liquid staking tokens into
$s$ RTS tokens, $s = b \frac{s_0}{b_0}$, and immediately sell them for
$b^{'}$ ATOM: $b^{'} \leq s \frac{b_0}{s_0}$.

Finally, the adversary shorts the liquid staking token, equivocates
and closes the short position.

The adversary has lost $qc$ ATOM, however the price of RTS
tokens has dropped.

Before the equivocation, the price of RTS tokens was $s_0/b_0$ but
after, it is:
\[
\frac{s_0 + s}{b_0} =
\frac{s_0 + s_0\frac{b}{b_0}}{b_0} =
\frac{s_0}{b_0} (1 + \frac{b}{b_0})
\]

The fraction $\frac{b}{b_0}$ is the slashing slippage.

% TODO: analyze rationality of the attack
% TODO: appendix? introduce discount of stasset -> asset in the market using loans
