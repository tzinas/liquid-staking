\import{./}{exempt-delegation-attack-diagram.tex}

\section{Exempt Delegations}

Exempt delegations (proposed in LSM~\cite{liquidity-staking-module})
are a mechanism to alleviate the Principal--Agent problem in liquid staking.
An exempt delegation amount $c$, measured in \asset, is associated
with each validator. It is a measure of the validator's trustworthiness.
The liquid staking protocol is now redesigned to impose restrictions
on how much of the protocol's pooled moneys can be delegated to a particular
validator based on the validator's exempt delegation.
The restriction is
parameterized by a factor $\phi$ (in practice, $\phi > 1$)
and is given by the inequality $b \leq \phi c$: Only up to $b$ \assets
are allowed to be delegated by the liquid staking protocol
to a validator with $c$ \assets in exempt delegations.

A new validator begins its lifecycle with $c = 0$. They can then
raise their own exempt delegation amount by locking aside a
chosen amount of \asset, and marking it as \emph{exempt}. Those
assets are delegated to the validator as usual. However,
the \emph{exempt}
marking means that those delegated assets cannot be part of the liquid
staking protocol pool, but must remain infungibly locked aside. Additionally,
these specially marked delegations are slashed\footnote{In the context
of Cosmos, the exempt delegation mechanism is planned to be applied at the consensus
layer by the Liquidity Staking Module (LSM)~\cite{liquidity-staking-module}.
When this mechanism is used, \emph{assets} are first delegated to a validator by
a delegator who obtains \emph{delegated assets}, marking them as exempt or non-exempt.
In the case of non-exempt delegated assets, these are then
\emph{tokenized} into \emph{LSM shares}, representations of delegated assets that
are minimally fungible (fungible among the other tokens that were delegated in the
same batch to the same validator). These tokenized
shares are subject to the exempt delegation constraint $b \leq \phi c$. The tokenized shares
can then be \emph{deposited} into the liquid staking protocol, which issues liquid staking
tokens (\stassets), as usual, in a process termed \emph{refungibilization}.
The protocol does not need to delegate further, as it can readily start reaping the
delegation rewards (as long as a relevant so-called \emph{LSM record} is also transferred
along with the tokenized shares).
It also does not need to perform further exempt delegation constraint checks.
When the user \emph{redeems} \stassets, the protocol may elect to give
back tokenized shares instead of \assets. Those can then be unwrapped into delegated assets,
that can afterwards be undelegated into \assets after the relevant unbonding
period expires.
Through this mechanism, the exempt delegation $c$ of a validator is
a \emph{shared} amount across potentially multiple liquid staking
protocols that opt to accept tokenized shares
instead of \assets directly. The \emph{intent} necessary for proportional
representation can be read by the liquid staking protocol by simply
looking at the LSM tokenized share records, and no separate voting is
necessary when entering the protocol. The factor $\phi$ is decided not by the
liquid staking protocols' governance, but by the governance of
the underlying chain. The slashing factor $q$ is applied directly
by the chain and not by the liquid staking protocol.
In the current LSM design, $q = p$.
If a liquid staking protocol participates in
multiple chains, the $\phi$ factors can be different in each chain.
In our exposition, we abstract out these implementation details to highlight
the economic issues at hand.}
at a potentially higher rate $q \geq p$. Exempt delegated assets cannot
be undelegated in a way that would cause a violation of the inequality
$b \leq \phi c$.

Delegators, whether wise or unwise, do not participate in exempt
delegations; instead, it is the validator who exempt delegates to
themselves (or someone who trusts the validator for extrinsic reasons).
This means that, in case of validator misbehavior, the exempt delegation
slashing $qc$ is a penalty that only affects the validator.

This raises
the cost of the attack described in the previous section. The
adversary must first, at time $t_0 < t_1$, exempt delegate a sufficient amount
$c \geq \frac{b}{\phi}$ \asset to $\mathcal{V}$ before she can liquid stake $b$ \asset.
Whereas the \stassets
corresponding to those $b$ \assets can be, as before, sold at $t_2$ to
separate the Agent from the Principal, the $c$ amount remains with the
Agent, holding her financially liable to misbehavior. After equivocation at $t_4$,
in addition to any other costs, the adversary loses $qc$ \asset. At the conclusion
of the attack, the adversary undelegates the remaining $(1 - q)c$ exempt delegation.

The attack may remain profitable despite exempt delegations.
The rational adversary should not waste any unnecessary resources on
$c$; therefore, she can set $c = \frac{b}{\phi}$. The profit of the attack now
becomes $b^* - b' - \frac{q}{\phi}b$.
The intuition for why exempt delegations protect the system is that,
for the adversary to profit from the short, she must cause a significant
shift in the price. The shift in the price is determined by the factor
$p\frac{b}{b_0}$, so the adversary aims for a large $b$. But because $b \leq \phi c$
must be respected, this incurs a large penalty $qc = \frac{q}{\phi}b$.

% TODO: analyze rationality of the attack
% TODO: appendix? introduce discount of stasset -> asset in the market using loans
