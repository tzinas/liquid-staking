\section{Exempt Delegations}

Exempt delegations (proposed in LSM~\cite{liquidity-staking-module})
are a mechanism to alleviate the Principal--Agent problem in liquid staking.
An exempt delegation amount $c$, measured in \asset, is associated
with each validator. It is a measure of the validator's trustworthiness.
The liquid staking protocol is now redesigned to impose restrictions
on how much of the pooled moneys can be delegated to a particular
validator based on their exempt delegation. The restriction is
parameterized by a factor $\phi$ (in practice, $\phi > 1$)
and is given by the inequality $b < \phi c$: Only up to $b$ \assets
are allowed to be delegated by the liquid staking protocol
to a validator with $c$ \asset in exempt delegations.

A new validator begins its lifecycle with $c = 0$. They can then
raise their own exempt delegation amount by locking aside a
chosen amount of \asset, and marking it as \emph{exempt}. Those
assets are delegated to the validator as usual. However,
the \emph{exempt}
marking means that those delegated assets cannot be part of the liquid
staking protocol pool, but must remain locked aside. Additionally,
these specially marked delegations are slashed\footnote{In the context
of Cosmos, the exempt delegation mechanism is applied at the consensus
layer by the Liquidity Staking Module (LSM)~\cite{liquidity-staking-module}.
When this mechanism is used, \emph{assets} are first delegated to a validator by
a delegator who obtains \emph{delegated assets}. These \emph{delegated assets} are then
\emph{tokenized} into \emph{LSM shares}, representations of delegated assets that
are minimally fungible (fungible among the other tokens that were delegated in the
same batch to the same validator). These tokenized
shares are subject the the exempt delegation constraint $b < \phi c$. The tokenized shares
can then be \emph{deposited} into the liquid staking protocol, which issues liquid staking
tokens (\stassets), as usual, in a process termed \emph{refungibilization}.
The protocol does not need to delegate further, as it can readily start reaping the
delegation rewards (as long as a relevant so-called \emph{LSM record} is also transferred
along with the tokenized shares).
It also does not need to perform further exempt delegation constraint checks.
When the user \emph{redeems} \stassets, the protocol may elect to give
back tokenized shares instead of \assets. Those can then be unwrapped into delegated assets,
that can afterwards be undelegated into \assets after the relevant unbonding
period expires.
Through this mechanism, the exempt delegation $c$ of a validator is
a \emph{shared} amount across potentially multiple liquid staking
protocols that opt to accept tokenized shares
instead of \assets directly. The \emph{intent} necessary for proportional
representation can be read by the liquid staking protocol by simply
looking at the LSM tokenized share records, and no separate voting is
necessary when entering the protocol. The factor $\phi$ is decided not by the
liquid staking protocols' governance, but by the governance of
the underlying chain. The slashing factor $q$ is applied directly
by the chain and not by the liquid staking protocol.
If a liquid staking protocol participates in
multiple chains, the $\phi$ factors can be different in each chain.
In our exposition, we abstract out these implementation details to highlight
the economic issues at hand.}
at a potentially higher rate $q \geq p$.

Delegators can choose to make an \emph{exempt delegation},
instead of a regular delegation, to a validator.
These exempt delegations, as their name suggests,
are \emph{exempt} from tokenization,
and are akin to self-delegation of older systems.
In case of validator misbehavior, exempt delegations
are slashed at a larger proportion $0 < q \leq 1$ than regular delegations.
Therefore, exempt delegations act as a meter of the
validator's trustworthiness.

% TODO: $\phi$
% Let $\varphi$ be the exempt delegation factor of the protocol
% that determines the amount $b$ as follows: $b \leq \varphi c$.

The amount of \asset in the protocol's delegation pool that can be
delegated to a specific validator is limited by the amount of its
exempt delegations and the protocol's \emph{Exempt Delegation Factor} $\varphi$.
Hence, if validator $\mathcal{V}$ has $c$ \asset worth of exempt delegations,
the amount of \asset that can be delegated to $\mathcal{V}$ by the protocol
is $b \leq \varphi c$.

The attack described in the previous section can be modified to work
around exempt delegations.

Like before, the adversary obtains capital $b$ \asset and creates
validator $\mathcal{V}$.
However, $\mathcal{V}$ has no exempt delegations yet and cannot be delegated to
by the protocol.
Therefore, she additionally obtains capital $c = \frac{b}{\varphi}$ \asset
and at time $t_0 < t_1$ uses it to exempt delegate to $\mathcal{V}$.
Now, $\mathcal{V}$ has enough exempt delegations to be delegated to by the protocol.

Then, once again, the adversary delegates $b$ \asset at time $t_1$,
sells the acquired $s$ \stasset in the market at time $t_2$, shorts \stasset by taking a loan
of $z$ \stasset at time $t_3$, equivocates using $\mathcal{V}$ at time $t_4$.

% TODO: Clean up the extension of the attack using the c parameter below

The adversary has lost $qc$ ATOM, however the price of RTS
tokens has dropped.

% TODO: analyze rationality of the attack
% TODO: appendix? introduce discount of stasset -> asset in the market using loans
