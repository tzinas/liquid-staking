\section{Exempt Delegations}

Exempt delegations (proposed in LSM~\cite{liquidity-staking-module})
are a mechanism to alleviate
the Principal--Agent problem in liquid staking.
Delegators can choose to make an \emph{exempt delegation},
instead of a regular delegation, to a validator.
These exempt delegations, as their name suggests,
are \emph{exempt} from tokenization,
and are akin to self-delegation of older systems.
In case of validator misbehavior, exempt delegations
are slashed at a larger proportion $0 < q \leq 1$ than regular delegations.
Therefore, exempt delegations act as a meter of the
validator's trustworthiness.

The amount of liquid staking tokens that can be
minted for a specific validator is limited by the amount of
exempt delegations made to that validator and the \emph{Exempt Delegation Factor}.
Therefore, in order for a shorting attack on the liquid staking
token to be profitable for the adversary, the shorting profit would
have to be greater than the exempt delegations that were slashed.

A precise formula for calculating the exempt delegation factor can be
created by formalizing the above attack.

\subsection{Attack}

Before the attack is initiated, there are $s_0$ RTS tokens and $b_0$
staked ATOM reserves in the protocol.

The adversary would start of by creating a validator and making an
exempt delegating of $c$ ATOM to it.
Then make a regular delegation of $b$ ATOM to the validator and
immediately convers all the regular delegation shares to $b$ liquid
staking tokens.
Let $\varphi$ be the exempt delegation factor of the protocol
that determines the amount $b$ as follows: $b \leq \varphi c$.

Then, the adversary would convert all the $b$ liquid staking tokens into
$s$ RTS tokens, $s = b \frac{s_0}{b_0}$, and immediately sell them for
$b^{'}$ ATOM: $b^{'} \leq s \frac{b_0}{s_0}$.

Finally, the adversary shorts the liquid staking token, equivocates
and closes the short position.

Let $q$ be the percentage of exempt delegations that will be slashed after
the equivocation. The adversary has lost $qc$ ATOM, however the price of RTS
tokens has droped.

Before the equivocation, the price of RTS tokens was $s_0/b_0$ but
after, it is:
\[
\frac{s_0 + s}{b_0} =
\frac{s_0 + s_0\frac{b}{b_0}}{b_0} =
\frac{s_0}{b_0} (1 + \frac{b}{b_0})
\]

The fraction $\frac{b}{b_0}$ is the slashing slippage.
