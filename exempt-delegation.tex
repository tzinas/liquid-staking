\section{Exempt Delegations}

Exempt delegations (proposed in LSM~\cite{liquidity-staking-module})
are a mechanism to alleviate
the Principal--Agent problem in liquid staking.
Delegators can choose to make an \emph{exempt delegation},
instead of a regular delegation, to a validator.
These exempt delegations, as their name suggests,
are \emph{exempt} from tokenization,
and are akin to self-delegation of older systems.
In case of validator misbehavior, exempt delegations
are slashed at a larger proportion $0 < q \leq 1$ than regular delegations.
Therefore, exempt delegations act as a meter of the
validator's trustworthiness.

% TODO: \phi

The amount of liquid staking tokens that can be
minted for a specific validator is limited by the amount of
exempt delegations made to that validator and the \emph{Exempt Delegation Factor}.
Therefore, in order for a shorting attack on the liquid staking
token to be profitable for the adversary, the shorting profit would
have to be greater than the exempt delegations that were slashed.

A precise formula for calculating the exempt delegation factor can be
created by formalizing the above attack.
