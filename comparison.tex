\section{Comparison of Liquid Staking Systems}

There are multiple liquid staking protocol approaches. In this section we
define 4 main categories and examine them in the context of proportional
representation and fair punishment.

\textbf{Governance.}
In liquid staking protocols following the governance approach,
the majority of protocol participants decide where the protocol's
funds are delegated. This means that only the majority of participants
are represented and proportional representation is not achieved.

If the majority of participants are wise, no malicious validator is
delegated by the protocol. Hence, there will be no slashing and no
protocol participant is unfairly punished.
However, if the majority of participants are unwise, malicious validators
are delegated by the protocol. In case of slashing, all protocol participants
are punished, including wise protocol participants.
Hence, there is a 50\% participant \emph{wiseness threshold} for the protocol
to have fair punishment.

\textbf{Centralized.}
In centalized liquid staking protocols, a centralized entity
selects the validator set and/or decides where the protocol's funds
are delegated. A participant's validator of choice could be exluded from
the validator set and/or participants may not have the oportunity to signal
their delegation intent. Hence, proportional representation is not
achieved. In extreme cases, no participant is represented.

The centralized entity could be unwise. The validator set could only
consist of adversarial validators and the protocol's funds could be delegated
to an adversarial validator by the centalized entity. In case of slashing,
all protocol participants, including wise participants, are punished.
Hence, fair punishment is not achieved. In extreme cases, where all
protocol participants are wise, no participant is fairly punished.

Some protocols following the centralized approach claim various levels of
governance decentralization, but, even though the community might be consulted
on governance matters, there is a central key or committee of keys which takes
the final decision. If those keys become compromised, the community decisions
can be bypassed.

% Trade-offs: -proportional representation, -fair punishment

% Write about proportional representation and trade-offs

\textbf{Algorithmic}
Protocols following this approach use an algorithmic process to determine
where the protocol's funds are delegated. Validators could be selected based on
commission rate, self-delegations, value locked or simply at random.
However, protocol participants may have different delegation preferences from
the protocol. Again, proportional representation is not achieved and in
extreme cases, no participant is represented.

If an adversarial validator is delegated by the protocol, all participants,
including wise participants, are punished in case of slashing. Hence,
fair punishment is not achieved. In extreme cases, where all protocol
participants are wise, no participant is fairly punished.

\textbf{Proportional Representation.}
Liquid staking protocols with proportional representation delegate
protocol funds so that all participants are represented proportionally
to their stake. Fair punishment is not achieved, as indicated in
Section~\ref{sec:attack}.

% Other protocols follow a proportion representation approach.

% Trade-offs: +proportional representation, -fair punishment but exempt
% delegations help


% \textbf{Non-Fungibility}
% Trade-offs: +proportional representation, +fair punishment but lose
% fungibility


% Lido -> No proportional representation: Validators are chosen by the
% Lido Node Operator Sub-governance Group (LNOSG) and voted by the
% Lido DAO. Fair punishment: The simplistic attack can be performed
% and the price of stETH will drop. Misbehaving validators are penalised
% by the Lido DAO.

% TODO:

% Rocket pool -> No proportional representation: user funds are randomly
% assigned to validators. However, anyone can be a validator without approval (given
% that the network is not underutilized). No fair punishment:
% the simplistic attack can be performed and rETH price will drop. There is
% a mechanism to make the shorting attack unprofitable. Any slashed amount
% is covered by the validator's ETH and RPL collateral.

% Parallel Finance -> No proportional representation: Validators probably
% are proposed by Parellel Finance and voted by the governance.
% No Fair punishment: If a validator is slashed, the price of
% sDOT will drop. They offer fund insurance if purchased.
% They offer instant unstaking of DOT by charging a borrowing fee
% discussed in this paper.
%https://docs.parallel.fi/parallel-finance/#2.-validator-choosing-strategy
%https://docs.parallel.fi/parallel-finance/#4.-council

% Ankr -> No proportional representation: Validator set is centralized.
% Fair punishment: If a validator is slashed, the price of ankrETH will drop.

% Marinade -> No proportional representaion: There is an algorithm to
% determine the chosen validators through a score (no "big" validators are
% selected). An automated bot (centralized) makes the delegations/redelegations.
% Delegator's can't choose their validator.
% Fair punishment: The
% price mSOL will drop in case of validator slashing.
% Marinade also offers immediate unstaking of SOL by charging a fee.

% pStake -> No proportional representation: The validator set is
% selected based on performance (maybe there is a formula?).
% The addition to the validator set and delegations/redelegations are
% centralized. No Fair punishment.


\begin{table}[]
\begin{tabular}{c|c|c}
                                                                                        & \textbf{Represented} & \textbf{Fairly punished} \\ \hline
\textbf{Quicksilver}                                                                    & 100\%                & 0\%                      \\
\textbf{Lido}                                                                           & 50\%                 & 50\%                     \\
\begin{tabular}[c]{@{}c@{}}\textbf{LSM} (non-fungible)\end{tabular} & 100\%                & 100\%                    \\
\textbf{Rocket Pool}                                                                    & 0\%                  & 0\%                      \\
\textbf{Stride}                                                                         & $\infty$             & $\infty$
\end{tabular}
\end{table}
