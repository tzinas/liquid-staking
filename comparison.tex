\section{Comparison of Liquid Staking Systems}

There are multiple liquid staking protocol approaches. In this section we
compare such protocols in the context of proportional representation
and fair punishment, and discuss the trade-offs between the
two properties.

\textbf{Governance.}
Some protocols, like Lido~\cite{lido}, follow a governance approach
where the majority of participants decide where the funds are
delegated. This means that only 50\% of participants are
represented and proportional representation is not achieved.

If the majority of participants are
wise, the protocol is also wise. Hence, the actions of malicious validators
do not affect wise protocol participants' stake.
However, if the majority of participants are unwise, the protocol is
also unwise. Wise protocol participants are unfairly punished in case
of validator misbehavior.
In the first instance, the protocol has fair punishment. In the
second one, it does not. Hence, there is a 50\% participant
\emph{wiseness threshold} for the protocol to have fair punishment.

\textbf{Centralized.}
In other protocols, a centralized entity is responsible for
selecting where the protocol's funds are delegated.
Factors such as commission rate,
uptime, and stake are considered to select the validators.
Signaling delegation intent is impossible. Hence, no protocol
participant is guaranteed to be represented.

If the centralized entity is unwise, protocol funds are
delegated to malicious validators. In case a malicious validator in the
set is slashed, all protocol participants are punished. Even if all
protocol participants are wise, they are still punished.
Hence, there is no fair punishment is this kind of protocols.

The above protocols claim various levels of governance decentralization,
but, even though the community might be consulted on governance matter,
there is a central key or committee of keys which takes the final decision.
If those keys become compromised, the community decisions can be bypassed, and
the liquid staking protocol can be broken.

% Trade-offs: -proportional representation, -fair punishment

% Write about proportional representation and trade-offs

\textbf{Algorithmic}
Other protocols use an algorithmic process to determine
where the protocol funds are delegated. Rocket Pool~\cite{rocketpool} %TODO
allows any validator to join the validator set. Then,
validators are delegated randomly by the protocol.
This may leave protocol participants unrepresented if they are not in
agreement with the algorithm. In extreme cases, no protocol participant may be
represented.

If the protocol delegates to a malicious validator that later equivacates,
all protocol participants are punished. That includes wise participants. Thus,
fair punishment is not achieved.


\textbf{Proportional Representation.}
% Other protocols follow a proportion representation approach.

% Trade-offs: +proportional representation, -fair punishment but exempt
% delegations help


% \textbf{Non-Fungibility}
% Trade-offs: +proportional representation, +fair punishment but lose
% fungibility


% Lido -> No proportional representation: Validators are chosen by the
% Lido Node Operator Sub-governance Group (LNOSG) and voted by the
% Lido DAO. Fair punishment: The simplistic attack can be performed
% and the price of stETH will drop. Misbehaving validators are penalised
% by the Lido DAO.

% TODO:

% Rocket pool -> No proportional representation: user funds are randomly
% assigned to validators. However, anyone can be a validator without approval (given
% that the network is not underutilized). No fair punishment:
% the simplistic attack can be performed and rETH price will drop. There is
% a mechanism to make the shorting attack unprofitable. Any slashed amount
% is covered by the validator's ETH and RPL collateral.

% Parallel Finance -> No proportional representation: Validators probably
% are proposed by Parellel Finance and voted by the governance.
% No Fair punishment: If a validator is slashed, the price of
% sDOT will drop. They offer fund insurance if purchased.
% They offer instant unstaking of DOT by charging a borrowing fee
% discussed in this paper.
%https://docs.parallel.fi/parallel-finance/#2.-validator-choosing-strategy

% Ankr -> No proportional representation: Validator set is centralized.
% Fair punishment: If a validator is slashed, the price of ankrETH will drop.

% Marinade -> No proportional representaion: There is an algorithm to
% determine the chosen validators through a score (no "big" validators are
% selected). An automated bot (centralized) makes the delegations/redelegations.
% Delegator's can't choose their validator.
% Fair punishment: The
% price mSOL will drop in case of validator slashing.
% Marinade also offers immediate unstaking of SOL by charging a fee.

% pStake -> No proportional representation: The validator set is
% selected based on performance (maybe there is a formula?).
% The addition to the validator set and delegations/redelegations are
% centralized. No Fair punishment.


\begin{table}[]
\begin{tabular}{c|c|c}
                                                                                        & \textbf{Represented} & \textbf{Fairly punished} \\ \hline
\textbf{Quicksilver}                                                                    & 100\%                & 0\%                      \\
\textbf{Lido}                                                                           & 50\%                 & 50\%                     \\
\begin{tabular}[c]{@{}c@{}}\textbf{LSM} (non-fungible)\end{tabular} & 100\%                & 100\%                    \\
\textbf{Rocket Pool}                                                                    & 0\%                  & 0\%                      \\
\textbf{Stride}                                                                         & $\infty$             & $\infty$
\end{tabular}
\end{table}
