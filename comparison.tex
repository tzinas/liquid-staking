\section{Comparison of Liquid Staking Systems}

In this section we define four main electoral mechanisms for liquid staking
protocol governance, and examine them in the context of fair punishment.
Our taxonomy of electoral mechanisms concerns the representation of the liquid
staking protocol stakeholders, those economic participants who are delegating
their stake to the protocol. Our analysis does not concern the representation of
other interest groups such as the developers of the protocol, initial investors,
early adopters, and so on.

\noindent
\textbf{Dictatorships.}
In a \emph{dictatorial}\footnote{We adopt the term \emph{dictatorship} from
social choice theory. A \emph{dictatorship mechanism}~\cite{arrow} is a mechanism in which
the results of voting mirror a single pre-determined person's preferences, without
consideration of the other voters.}
liquid staking protocol, a central entity
selects the validator set or decides where the protocol's funds
are delegated.
Dictatorships may be governed by one signing key or a
committee of keys. The distinguishing aspect of a committee-based dictatorship
is that this committee is not necessarily related to the actual economic stakeholders
of the liquid staking protocol.
Proportional representation is not
achieved.

If the central entity is unwise, the validator set
consists of adversarial validators and the protocol's funds are delegated
to them. When the validator misbehaves,
all protocol participants, including wise participants, are punished.
Hence, fair punishment is not achieved.

Some protocols following the dictatorial approach claim various levels of
governance decentralization~\cite{???} and democracy~\cite{parallel}. % TODO(tzinas): Add more citations
Even though the community might vote (off- or on-chain)
on governance matters, there is a central party or committee of parties
which promise to transfer this community decision to the protocol.
The final on-chain decision
is made by having the central key or committee of keys sign an on-chain transaction.
If those keys become compromised, or if the holders of the keys disagree with the community
decision, there is always a technical mechanism to bypass it.

\noindent
\textbf{Majoritarian.}
In majoritarian liquid staking protocols,
the majority of protocol participants decide where the protocol's
funds are delegated. This means that only the majority of participants
are represented and proportional representation is not achieved.

If the majority of participants are wise, no malicious validator is
selected for delegation by the protocol. Hence, there will be no slashing and no
protocol participant is unfairly punished.
However, if the majority of participants are unwise, malicious validators
are selection for delegation. In case of slashing, all protocol participants
are punished, including wise protocol participants.
Hence, there is, at best, a 50\% participant \emph{wisdom threshold} for the protocol
to have fair punishment\footnote{The wisdom threshold can never be below 50\%
even when voting tresholds are higher or lower than 50\%.
% Safety issue:
If the voting threshold is below 50\% and the wise participants constitute
50\% of the population, a small unwise group
can delegate to an adversrial validator and cause unfair punishment.
% Liveness issue:
If the voting threshold is above 50\% and the wise participants constitute
50\% of the population, they have insufficient voting power to outvote a
validator who has turned adversarial.}

% wisdom_threshold = max(voting_threshold, 1 - voting_threshold)

\noindent
\textbf{Proportional Representation.}
Liquid staking protocols with proportional representation delegate
protocol funds so that all participants are represented proportionally
to their stake. Fair punishment is not achieved, as indicated in
Section~\ref{sec:attack}.

% Trade-offs: -proportional representation, -fair punishment

% Write about proportional representation and trade-offs

\noindent
\textbf{Algorithmic.}
Protocols following this approach use an algorithmic process to determine
where the protocol's funds are delegated. Validators could be selected based on
commission rate, self-delegations, value locked or simply at random.
However, protocol participants may have different delegation preferences from
the protocol. Again, proportional representation is not achieved and in
extreme cases, no participant is represented.

If an adversarial validator is delegated by the protocol, all participants,
including wise participants, are punished in case of slashing. Hence,
fair punishment is not achieved. In extreme cases, where all protocol
participants are wise, no participant is fairly punished.

% Other protocols follow a proportion representation approach.

% Trade-offs: +proportional representation, -fair punishment but exempt
% delegations help


% \textbf{Non-Fungibility}
% Trade-offs: +proportional representation, +fair punishment but lose
% fungibility


% Lido -> No proportional representation: Validators are chosen by the
% Lido Node Operator Sub-governance Group (LNOSG) and voted by the
% Lido DAO. Fair punishment: The simplistic attack can be performed
% and the price of stETH will drop. Misbehaving validators are penalised
% by the Lido DAO.

% TODO:

% Rocket pool -> No proportional representation: user funds are randomly
% assigned to validators. However, anyone can be a validator without approval (given
% that the network is not underutilized). No fair punishment:
% the simplistic attack can be performed and rETH price will drop. There is
% a mechanism to make the shorting attack unprofitable. Any slashed amount
% is covered by the validator's ETH and RPL collateral.
% We need to speak about the difference between exempt delegations and
% Rocket's mechanism.

% Parallel Finance -> No proportional representation: Validators probably
% are proposed by Parellel Finance and voted by the governance.
% No Fair punishment: If a validator is slashed, the price of
% sDOT will drop. They offer fund insurance if purchased.
% They offer instant unstaking of DOT by charging a borrowing fee
% discussed in this paper.
%https://docs.parallel.fi/parallel-finance/#2.-validator-choosing-strategy
%https://docs.parallel.fi/parallel-finance/#4.-council

% Ankr -> No proportional representation: Validator set is centralized.
% Fair punishment: If a validator is slashed, the price of ankrETH will drop.

% Marinade -> No proportional representaion: There is an algorithm to
% determine the chosen validators through a score (no "big" validators are
% selected). An automated bot (centralized) makes the delegations/redelegations.
% Delegator's can't choose their validator.
% Fair punishment: The
% price mSOL will drop in case of validator slashing.
% Marinade also offers immediate unstaking of SOL by charging a fee.

% pStake -> No proportional representation: The validator set is
% selected based on performance (maybe there is a formula?).
% The addition to the validator set and delegations/redelegations are
% centralized. No Fair punishment.

% \footnote{Stride currently have
% taken a centralized decision, but plan to make future decisions by voting
% among liquid stake holders~\cite{stride-validators}. Lido and Persistence
% use a separate governance token~\cite{lido-validators,persistence-validators} for
% such decision making in place of stakeholders.}

% https://learn.bybit.com/altcoins/what-is-lido-dao-token-ldo/
% ``The founding members have 64% of the tokens, locked for one year and vested from there.''

\begin{table}[]
\begin{tabular}{c|c|c}
                                                                                        & \textbf{Represented} & \textbf{Fairly punished} \\ \hline
\textbf{Quicksilver}                                                                    & 100\%                & 0\%                      \\
\textbf{Lido}                                                                           & 50\%                 & 50\%                     \\
\begin{tabular}[c]{@{}c@{}}\textbf{LSM} (non-fungible)\end{tabular} & 100\%                & 100\%                    \\
\textbf{Rocket Pool}                                                                    & 0\%                  & 0\%                      \\
\textbf{Stride}                                                                         & $\infty$             & $\infty$
\end{tabular}
\end{table}
