\section{Comparison of Liquid Staking Systems}

Liquid staking protocols take different approaches to try balancing
proportional representation and fair punishment. In this section we
compare such protocols and discuss the trade-offs between them.

\textbf{Governance.}
The choice of where the protocol delegates its funds depends on the
protocol itself. Some protocols, like Lido~\cite{lido}, follow a governance approach
where the majority of participants decide where the funds are
delegated. If the majority of participants are
wise, the protocol is also wise. Hence, the actions of a malicious validator
will not affect wise protocol participants' stake.
However, if the majority of participants are unwise, the protocol is
also unwise. Wise protocol participants will be unfairly punished in case
of validator misbehavior.
In the first instance, the protocol has fair punishment. In the
second one, it does not.

% Write about proportional representation and trade-offs (balanced)

\textbf{Proportional Representation.}
% Other protocols follow a proportion representation approach.

% Trade-offs: +proportional representation, -fair punishment but exempt
% delegations help

\textbf{Centralized.}
% Trade-offs: -proportional representation, -fair punishment

\textbf{Non-Fungibility}
% Trade-offs: +proportional representation, +fair punishment but lose
% fungibility


% Lido -> No proportional representation: Validators are chosen by the
% Lido Node Operator Sub-governance Group (LNOSG) and voted by the
% Lido DAO. Fair punishment: The simplistic attack can be performed
% and the price of stETH will drop. Misbehaving validators are penalised
% by the Lido DAO.

% Rocket pool -> No proportional representation: user funds are randomly
% assigned to validators. However, anyone can be a validator without approval (given
% that the network is not underutilized). No fair punishment:
% the simplistic attack can be performed and rETH price will drop. There is
% a mechanism to make the shorting attack unprofitable. Any slashed amount
% is covered by the validator's ETH and RPL collateral.

% Parallel Finance -> No proportional representation: Validators probably
% are proposed by Parellel Finance and voted by the governance.
% No Fair punishment: If a validator is slashed, the price of
% sDOT will drop. They offer fund insurance if purchased.
% They offer instant unstaking of DOT by charging a borrowing fee
% discussed in this paper.
%https://docs.parallel.fi/parallel-finance/#2.-validator-choosing-strategy

% Ankr -> No proportional representation: Validator set is centralized.
% Fair punishment: If a validator is slashed, the price of ankrETH will drop.

% Marinade -> No proportional representaion: There is an algorithm to
% determine the chosen validators through a score (no "big" validators are
% selected). Delegator's can't choose their validator Fair punishment: The
% price mSOL will drop in case of validator slashing.
% Marinade also offers immediate unstaking of SOL by charging a fee.

\begin{table}[]
\begin{tabular}{c|c|c}
                                                                                        & \textbf{Represented} & \textbf{Fairly punished} \\ \hline
\textbf{Quicksilver}                                                                    & 100\%                & 0\%                      \\
\textbf{Lido}                                                                           & 50\%                 & 50\%                     \\
\begin{tabular}[c]{@{}c@{}}\textbf{LSM} (non-fungible)\end{tabular} & 100\%                & 100\%                    \\
\textbf{Rocket Pool}                                                                    & 0\%                  & 0\%                      \\
\textbf{Stride}                                                                         & $\infty$             & $\infty$
\end{tabular}
\end{table}
