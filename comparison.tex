\section{Comparison of Liquid Staking Systems}

Proportional Representation and Fair Punishment are two desirable
properties of a liquid staking protocol.
Proportional representation enables decentralization by allowing every
participant to decide where their stake is delegated.
Fair punishment ensures that only unwise participants get punished for
their delegation decisions.
As of today, there are multiple protocols that take different approaches
to try to balance these two properties. In this section we will
compare such protocols and discuss the trade-offs between them.

\textbf{Governance.}
The choice of where the protocol delegates its funds depends on the
protocol itself. Some protocols, like Lido~\cite{lido}, follow a governance approach
where the majority of participants decide where the funds are
delegated. If the majority of participants are
wise, the protocol is also wise. Hence, the actions of a malicious validator
will not affect wise protocol participants' stake.
However, if the majority of participants are unwise, the protocol is
also unwise. Wise protocol participants will be unfairly punished in case
of validator misbehavior.
In the first instance, the protocol has fair punishment. In the
second one, it does not.

% Write about proportional representation and trade-offs (balanced)

\textbf{Proportional Representation.}
% Other protocols follow a proportion representation approach.

% Trade-offs: +proportional representation, -fair punishment but exempt
% delegations help

\textbf{Centralized.}
% Trade-offs: -proportional representation, -fair punishment

\textbf{Non-Fungibility}
% Trade-offs: +proportional representation, +fair punishment but lose
% fungibility

\begin{table}[t]
\centering
\begin{tabular}{c|ccc}
                     & {\color[HTML]{000000} \textbf{Fungibility}}        & \textbf{\begin{tabular}[c]{@{}c@{}}Proportional representation\\ stake threshold\end{tabular}} & \textbf{\begin{tabular}[c]{@{}c@{}}Fair punishment \\ wisdom threshold\end{tabular}} \\ \hline
\textbf{Quicksilver} & \cellcolor[HTML]{B0D8A4}{\color[HTML]{000000} YES} & \cellcolor[HTML]{B0D8A4}{\color[HTML]{000000} 0\%}                                             & \cellcolor[HTML]{E84258}{\color[HTML]{000000} 100\%}                                 \\
\textbf{Lido}        & \cellcolor[HTML]{B0D8A4}{\color[HTML]{000000} YES} & \cellcolor[HTML]{FEE191}50\%                                                                   & \cellcolor[HTML]{FEE191}{\color[HTML]{000000} 50\%}                                  \\
\textbf{LSM}         & \cellcolor[HTML]{E84258}{\color[HTML]{000000} NO}  & \cellcolor[HTML]{B0D8A4}{\color[HTML]{000000} 0\%}                                             & \cellcolor[HTML]{B0D8A4}{\color[HTML]{000000} 0\%}                                   \\
\textbf{Stride}      & \cellcolor[HTML]{B0D8A4}{\color[HTML]{000000} YES} & \cellcolor[HTML]{8281A0}{\color[HTML]{FFFFFF} $\infty$}                                        & \cellcolor[HTML]{8281A0}{\color[HTML]{FFFFFF} $\infty$}
\end{tabular}
\end{table}
