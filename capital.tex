\noindent
\textbf{Realizing profits in USD.}
If the attacker wants to do bookkeeping in a more stable reference currency,
such as USD, the attack is still profitable. The attacker begins by buying
\asset for USD. At the end of the attack, the attacker sells \asset for USD.
Because the attack concerns a particular liquid staking protocol, and
not the whole \asset network, the price of \asset will likely not
be significantly affected by the attack at all.
This attack decreases the market confidence in \stasset,
but not in \asset.
In fact, because
the attack causes slashing of \asset, the supply of \asset is decreased
and the price of \asset with respect to the reference currency may
even increase.
Lastly, any price fluctuations of \asset with respect to USD will likely be
minor, as the attack has a short duration of a couple of seconds.

\noindent
\textbf{The Market price of \stasset.}\label{sec:stasset-price}
Let us consider the price $k$ of \stasset denominated in \asset in the market.
Because the option always exists to mint at a rate of $\frac{s_0}{b_0}$ by
depositing, the price of \stasset denominated in \asset in a perfectly
efficient market is $\frac{b_0}{s_0}$ at maximum. Otherwise, no
rational buyer would use the market. Hence, the market rate is
$k \leq \frac{b_0}{s_0}$.

There are two options to convert $s$ \stasset to \asset: either sell
at the market rate to obtain $b = k s$ \asset, or use the redemption mechanism.
Using the redemption mechanism, the \assets become available after
time $\delta$.
Initially, using $s$ \stasset, a redemption is made of $b' = s \frac{b_0}{s_0}$
delegated assets. To get $b$ \asset immediately (and avoid having to wait
for the unbonding period), a loan of $b$ \asset is taken~\cite[p.~13]{liquid-staking-report} and
repayed after duration $\delta$.
The amount of \asset that needs to be paid back,
including principal and interest, is $((1 + \rasset)^\delta + \betaasset)b$ \asset.
We set this amount to be equal to $b'$, the amount of \assets that will be
unbonded after $\delta$ time. Solving for $b$, we get
$b = s \frac{b_0}{s_0 ((1 + \rasset)^\delta + \betaasset)}$.
Therefore, in an efficient market $k \geq \frac{b_0}{s_0 ((1 + \rasset)^\delta + \betaasset)}$.
We deduce that the bounds for an efficient market of \asset and \stasset are

\[
  \frac{b_0}{s_0 ((1 + \rasset)^\delta + \betaasset)} \leq k \leq \frac{b_0}{s_0}\,.
\]

The longer the duration $\delta$, the larger the potential price deviation
(c.f. the empirical analysis in
\emph{Liquid Staking: Basis Determinants and Price Discovery}~\cite{scharnowski2022liquid}).
The process of the loan is automated in some protocols~\cite[\S5]{parallel}\cite{marinade-matching}.

It is also possible to use the principle of \emph{remittances}
(matching)~\cite[\S5]{parallel}\cite{marinade-matching} to match depositing and
redeeming parties, so that the redeeming party does not have to incur any unbonding delay.
If the redeemed amounts exceed the deposited amounts, some amounts will necessarily incur
a delay. However, the above bounds may effectively be tighter than $\delta$ due to
a shorter effective unbonding duration.

% TODO (e-print): Re-state the attack formulae using the corrected pricing of stasset