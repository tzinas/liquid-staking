\section{The Principal--Agent Problem}
In the Proof-of-Stake systems described in the previous sections,
principals grant permission of their funds to validators so they can
participate in consensus on their behalf.
Hence, principals have ownership of the stake, but
validators have control over it.
The stakeholders rely on validators to act according to their
best interest: Stay online and follow the protocol.
However, validators (the agent) may have extrinsic motivation to
misbehave and plot against principals. This creates
a conflict of interest that is known as the Principal--Agent problem.

For example, a malicious validator can equivocate, which can cause
the principal's funds to be slashed. If the malicious validator has
limited self-delegation and no reputation to lose, the validator
may be able to profit from the principal's loss. However, a validator
with a larger self-delegation will be affected by the slashing themselves.
This is why self-delegation is considered to offer a layer of protection
against the Principal--Agent problem.

In traditional staking protocols, principals have the responsibility to
delegate their funds \emph{wisely}.
When a malicious validator misbehaves, only the stake of unwise principals is
slashed. No wise principal gets \emph{unfairly punished}.

% TODO(e-print): Formalize fair punishment
\begin{definition}[Fair Punishment]
    A staking protocol has Fair Punishment if no wise principal's
    stake gets slashed as a result of a malicious validator's actions.
\end{definition}

With the introduction of liquid staking protocols, the delegator
now becomes the protocol itself. If at least one of the validators
delegated to by the protocol is malicious, the protocol is
considered unwise. This means that a malicious validator's actions
will cause slashing of the protocol's delegations.
All of the protocol's delegations are fungible with one another.
Hence, the entire market built on top of the protocol will be affected by
the slashing. It is clear that the Principal--Agent problem is exacerbated.
There is now a new incentive for validators to equivocate and short the
protocol's market for profit.

%\begin{itemize}
%    \item More nuanced model than honest/adversarial: stakeholders are not honest or adversarial, but can make ``right'' or ``wrong'' choice; validators are honest and adversarial.
%    \item Fair punishment: You should not be slashed if you make the ``right'' choice.
%    \item Wise and unwise model
%    \item Why the problem is exacerbated (with example from governance)
%\end{itemize}
