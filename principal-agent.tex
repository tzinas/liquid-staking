\section{The Principal--Agent Problem}
In the Proof-of-Stake systems described in the previous sections,
delegators grant permission of their funds to validators so they can
participate in consensus on their behalf.
Hence, delegators have ownership of the stake, but
validators have control over it.
The stakeholders rely on validators to act according to their
best interest: Stay online and follow the protocol.
However, validators (the Agent) may have extrinsic motivation to
misbehave and plot against delegators (the Principal). This creates
a conflict of interest that is known as the Principal--Agent problem.

For example, a malicious validator can equivocate, which can cause
the delegator's funds to be slashed. If the malicious validator has
limited self-delegation and no reputation to lose, the validator
may be able to profit from the delegator's loss. However, a validator
with a larger self-delegation will be affected by the slashing themselves.
This is why self-delegation is considered to offer a layer of protection
against the Principal--Agent problem.

In traditional staking protocols, delegators have the responsibility to
delegate their funds \emph{wisely}.
They are not considered to be honest or adversarial, like validators.
They can just make ``right'' or ``wrong'' choices.
A delegator that delegates to a malicious validator is considered unwise,
whereas a delegator that delegates only to honest validators is
considered wise.

When a malicious validator misbehaves, only the stake of unwise delegators is slashed.
This is because no wise delegator would have delegated to a malicious
validator. Hence, no wise delegator gets punished as a result of
the malicious validator's misbehavior. We consider protocols with
this property to have \emph{fair punishment}.

\begin{definition}[Fair Punishment]
    A staking protocol has Fair Punishment if no wise
    delegator's stake gets slashed as a result of a malicious validator's
    actions.
\end{definition}

With the introduction of liquid staking protocols, the delegator
now becomes the protocol itself. If at least one of the validators
delegated to by the protocol is malicious, the protocol is
considered unwise. This means that a malicious validator's actions
will cause slashing of the protocol's delegations.
All of the protocol's delegations are fungible with one another.
Hence, the entire market built on top of the protocol will be affected by
the slashing. It is clear that the Principal--Agent problem is exacerbated.
There is now a new incentive for validators to equivocate and short the
protocol's market for profit.

%\begin{itemize}
%    \item More nuanced model than honest/adversarial: stakeholders are not honest or adversarial, but can make ``right'' or ``wrong'' choice; validators are honest and adversarial.
%    \item Fair punishment: You should not be slashed if you make the ``right'' choice.
%    \item Wise and unwise model
%    \item Why the problem is exacerbated (with example from governance)
%\end{itemize}
