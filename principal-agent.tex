\pdfbookmark[section]{The Principal--Agent Problem}{The Principal--Agent Problem}
\section{The Principal--Agent Problem}
In the proof-of-stake systems described in the previous sections,
principals grant permission of their funds to validators so they can
participate in consensus on their behalf.
Hence, principals have ownership of the stake, but
validators have control over it.
The stakeholders rely on validators to act according to their
best interest: Stay online and follow the protocol.
However, validators (the agents) may have extrinsic motivation to
misbehave and plot against principals. This creates
a conflict of interest that is known as the Principal--Agent problem.

For example, a malicious validator can equivocate, which causes
the principal's funds to be slashed. If the malicious validator has
limited self-delegation and no reputation to lose, the validator
may be able to profit from the principal's loss. However, a validator
with a larger self-delegation will themselves be affected by the slashing.
This is why self-delegation offers a layer of protection
against the Principal--Agent problem.

In traditional staking protocols, principals have the responsibility to
delegate their funds \emph{wisely}.
When a malicious validator misbehaves, only the stake of unwise principals is
slashed. No wise principal gets \emph{unfairly punished}.

% TODO(e-print): Formalize fair punishment
\begin{definition}[Fair Punishment]
    A staking protocol has Fair Punishment if no wise principal's
    stake gets slashed as a result of a malicious validator's actions.
\end{definition}

With the introduction of liquid staking protocols, the principal
is no longer directly delegating to the validator of their choice.
Instead, the protocol is now responsible for the delegation process
and stake allocation to validators.
Although the principal can express their delegation wish, ultimately
the protocol decides where funds are delegated based on its
\emph{delegation strategy}. All liquid staking tokens are fungible,
hence all validators delegated to by
the protocol become agents for all principals.
The Principal--Agent problem is exacerbated.
The principal's funds are now effectively delegated to validators
he has not necessarily chosen, some with mischievous intentions.