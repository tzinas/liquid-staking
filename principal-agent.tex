\section{The Principal--Agent Problem}
In the Proof-of-Stake systems described in the previous sections,
delegators grant permission of their funds to validators so they can
participate in consensus on their behalf.
Hence, delegators have ownership of the stake, but
validators have control over it.
The stakeholders rely on validators to act according to their
best interest: Stay online and follow the protocol.
However, validators (the Agent) may have extrinsic motivation to
misbehave and plot against delegators (the Principal). This creates
a conflict of interest that is known as the Principal--Agent problem.

For example, a malicious validator can equivocate, which can cause
the delegator's funds to be slashed. If the malicious validator has
limited self-delegation and no reputation to lose, the validator
may be able to profit from the delegator's loss. However, a validator
with a larger self-delegation will be affected by the slashing themselves.
This is why self-delegation is considered to offer a layer of protection
against the Principal--Agent problem.

\begin{itemize}
    \item More nuanced model than honest/adversarial: stakeholders are not honest or adversarial, but can make ``right'' or ``wrong'' choice; validators are honest and adversarial.
    \item Fair punishment: You should not be slashed if you make the ``right'' choice.
    \item Wise and unwise model
    \item Why the problem is exacerbated (with example from governance)
\end{itemize}
