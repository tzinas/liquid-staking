\section{Introduction}

Staking is a mechanism used in proof-of-stake (PoS)
blockchains~\cite{2018tendermint,buchman2016tendermint,ouroboros,algorand,casper}
to ensure security. Validators participating in PoS
\emph{bond} their stake, locking it up for a certain period of time,
in exchange for interest in the form of rewards.
This locked-up stake is \emph{slashed} in case of validator misbehavior
such as equivocations.
Stakeholders can \emph{delegate} their stake to \emph{validators} to also
earn rewards. If the validator a delegator delegates to misbehaves,
the funds of the delegator are also slashed. This introduces a
\emph{Principal--Agent} problem~\cite{jensen1976,wealthofnations}
in which the actions of the validator (the \emph{Agent})
affect the capital of the delegator (the \emph{Principal}).

However, staked assets are \emph{illiquid} and cannot be used for
other purposes such as in DeFi applications~\cite{defi-sok}
because they are locked up.
\emph{Liquid staking}~\cite{liquid-staking-report}
is an attempt to solve this problem by issuing
\emph{token representations} of the staked assets that can be freely traded
and utilized by stakeholders elsewhere in the blockchain ecosystem.

Liquid staking token representations are most valuable when they are
\emph{fungible}. However, this fungibility exacerbates the Principal--Agent
problem of delegated stake.
A liquid staking system \emph{pools together} stakes from different participants.
The question then arises of whom to delegate these pooled moneys to.
If the system has \emph{proportional representation}, every pool participant
can decide whom to delegate to \emph{in proportion} to their contributed shares.
The crux of the issue arises from the fact
that a malicious pool participant can choose to secretely delegate to
a colluding validator who then equivocates. This causes a portion
of the pooled money to be slashed, affecting every pool participant
in proportion to their contributed stake, even if they never made
any \emph{unwise} delegation decisions, in a situation of
\emph{unfair punishment}. This causes a drop in the price of the
liquid staking tokens. A rational attacker can profit from this price
drop by shorting the token.

\noindent
\textbf{Our contributions.} Our contributions in this paper are as follows:
\begin{enumerate}
    \item We introduce two desirable properties in the context of Liquid Staking: \emph{Proportional Representation} and \emph{Fair Punishment}.
    \item We showcase the Principal--Agent problem in the Liquid Staking setting by describing a concrete attack leveraging it.
    \item We give a precise description of the market conditions that enable this attack, and a formula for liquid staking system configuration which can avoid it.
\end{enumerate}

\import{./}{related.tex}
