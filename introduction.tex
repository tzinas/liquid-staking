\section{Introduction}

Staking is a mechanism used in proof-of-stake (PoS) blockchain
systems~\cite{2018tendermint, buchman2016tendermint,
ouroboros, algorand, casper} to ensure the security and
integrity of the network.
Users can delegate their tokens to validators, to be used as
collateral in the consensus validation process, earning rewards
in return.

However, staked assets are illiquid and cannot be used for
other purposes such as in DeFi applications~\cite{defi-sok}.
\emph{Liquid staking} tries to solve this problem by issuing
token representations of the staked assets, that can be utilized
by stakeholders elsewhere in the blockchain ecosystem.

\begin{itemize}
    \item Staking, Delegation
    \item Liquid Staking
        \begin{itemize}
            \item Staking APY + DeFi APY
            \item Liquid Stake VS Underlying price difference 
            \item Fungibility
        \end{itemize}
\end{itemize}

\noindent
\textbf{Our contributions.} Our contributions in this paper are as follows:
\begin{enumerate}
    \item We describe and systematize the desirable properties of Liquid Staking systems.
    \item We showcase the Principal--Agent problem in the Liquid Staking setting and describe a concrete attack leveraging it.
    \item We give a precise formula of Exempt Delegations which resolves the Principal--Agent problem.
\end{enumerate}

Quicksilver: ~\cite{quicksilver}
Stride: ~\cite{stride}
Lido: ~\cite{lido}
Persistence: ~\cite{persistence}
Tendermint: ~\cite{2018tendermint, buchman2016tendermint}
Cosmos SDK: ~\cite{cosmossdk}
Ouroboros: ~\cite{ouroboros}
Ouroboros Genesis: ~\cite{ouroboros-genesis}
Ouroboros Praos: ~\cite{praos}
Algorand: ~\cite{algorand}
Snow White: ~\cite{DBLP:journals/iacr/BentovPS16a}
PoS Ethereum: ~\cite{casper,buterin2020combining,sompolinsky2015secure,kiayias2017trees}
Price difference: ~\cite{scharnowski2022liquid}
DeFi SOK: ~\cite{defi-sok}
Principal--Agent problem: ~\cite{jensen1976,wealthofnations}
Eigenlayer~\cite{eigenlayer}