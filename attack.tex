\subsection{Attack}

We now describe an attack an adversary can conduct which leverages the
Principal--Agent problem of liquid staking.

\noindent
\textbf{Attack overview.}
The core idea is that the adversary can create her own validator
which she will collude with.
Even though no one else will choose to delegate to
this validator, the adversary can delegate to this validator herself.
The adversary begins the attack with some initial capital of \asset, which
can be her own capital or borrowed capital. The goal is to increase her
\asset holdings. She uses a portion of this capital to meet the exempt
delegations of her own validator. Then, she liquid stakes some other
portion of her capital, signaling delegation intent to her own validator
and receiving \stasset in return. She sells these tokens in the open
market in return for \asset. Next, she shorts \stasset and equivocates
using her own validator. This causes a slashing of parts of the \asset
holdings of the liquid staking protocol, and therefore a drop in the
price of \stasset as denominated in \asset. Lastly, the adversary
closes her short position and realizes her profits.

Due to the liquid staking protocol respecting proportional
representation, any signalling of the adversary.

Before the attack is initiated, there are $s_0$ RTS tokens and $b_0$
staked ATOM reserves in the protocol.

The adversary would start off by creating a validator and making an
exempt delegating of $c$ ATOM to it.
Then make a regular delegation of $b$ ATOM to the validator and
immediately convert all the regular delegation shares to $b$ liquid
staking tokens.
Let $\varphi$ be the exempt delegation factor of the protocol
that determines the amount $b$ as follows: $b \leq \varphi c$.

Then, the adversary would convert all the $b$ liquid staking tokens into
$s$ RTS tokens, $s = b \frac{s_0}{b_0}$, and immediately sell them for
$b^{'}$ ATOM: $b^{'} \leq s \frac{b_0}{s_0}$.

Finally, the adversary shorts the liquid staking token, equivocates
and closes the short position.

Let $q$ be the percentage of exempt delegations that will be slashed after
the equivocation. The adversary has lost $qc$ ATOM, however the price of RTS
tokens has dropped.

Before the equivocation, the price of RTS tokens was $s_0/b_0$ but
after, it is:
\[
\frac{s_0 + s}{b_0} =
\frac{s_0 + s_0\frac{b}{b_0}}{b_0} =
\frac{s_0}{b_0} (1 + \frac{b}{b_0})
\]

The fraction $\frac{b}{b_0}$ is the slashing slippage.

\noindent
\textbf{Realizing profits in USD.}
If the attacker wants to do bookkeeping in a more stable reference currency,
such as USD, the attack is still profitable. The attacker begins by buying
\asset for USD. At the end of the attack, the attacker sells \asset for USD.
Because the attack concerns a particular liquid staking protocol, and
not the whole \asset network, the price of \asset will likely not
be significantly affected by the attack at all.
This attack decreases the market confidence in \stasset,
but not in \asset.
In fact, because
the attack causes slashing of \asset, the supply of \asset is decreased
and the price of \asset with respect to the reference currency may
even increase. 
Lastly, any price fluctuations of \asset with respect USD will likely be
minor, as the attack has a short duration of a couple of seconds.