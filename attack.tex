\section{Attack}

We now describe an attack an adversary can conduct which leverages the
Principal--Agent problem of liquid staking. First, we observe that
fair punishment in the class of liquid staking protocols we are concerned
about is impossible.

\begin{claim}
Any \emph{fungible} liquid staking protocol with \emph{Proportional Representation}
deployed over any proof-of-stake consensus protocol with \emph{unbonding period}
$\delta > 0$ which slashes equivocating validators by a rate of $p > 0$
cannot have \emph{fair punishment}.
\end{claim}

\noindent
\textbf{A first attack attempt.}
To see why the above claim holds, consider the following simplistic attack.
Let $b_0$ be the amount of delegated \asset in the protocol's delegation pool,
and $s_0$ be the total amount of \stasset tokens outstanding
before the attack commences. The initial quoted price of \stasset
is $\frac{b_0}{s_0}$.

Initially,
the adversary $\mathcal{A}$ creates\footnote{To create this validator, she uses a fresh identity
to suppress potential suspicions. While most validators
have a real-world presence and can be held legally
accountable~\cite{liquid-staking-report}, this validator is pseudonymous.}
a new validator $\mathcal{V}$ under her control.
We do not require any of the existing protocol participants to delegate
to this validator for the attack to work, i.e., we assume all participants
are wise and all other validators are honest.
At time $t_2$, the adversary deposits $b$ \asset to the protocol,
signalling delegation intent to $\mathcal{V}$.
Due to Proportional Representation,
the protocol respects this intent and delegates $b$ \asset to $\mathcal{V}$.
The protocol now holds $b$ delegated \asset to $\mathcal{V}$.
% Reduntaunt? ^^
Through this deposit, the adversary obtains
$s = \frac{s_0}{b_0} b$ \stasset, and the quoted price remains
$\frac{b_0 + b}{s_0 + s} = \frac{b_0}{s_0}$.
Lastly, at time $t_4 > t_2$, the validator
$\mathcal{V}$ equivocates. This causes a proportion $p$ of
the capital $b$ to be slashed.
However the amount of \stasset circulating in the
market remains $s_0 + s = s_0 + b\frac{s_0}{b_0}$.
The new quoted price is now
$\frac{b_0 + (1 - p)b}{s_0 + s} = \frac{b_0}{s_0}(1 - p\frac{b}{b_0 + b}) < \frac{b_0}{s_0}$.

\import{./}{simple-attack-diagram.tex}

% \begin{align*}
%  &\frac{b_0 + (1 - p)b}{s_0 + s}\\
% =&\frac{b_0 + (1 - p)b}{s_0}\frac{1}{1 + \frac{b}{b_0}}\\
% =&\frac{b_0}{s_0}\frac{1 + (1 - p)\frac{b}{b_0}}{1 + \frac{b}{b_0}}\\
% =&\frac{b_0}{s_0}\frac{1 + \frac{b}{b_0} - p\frac{b}{b_0}}{1 + \frac{b}{b_0}}\\
% =&\frac{b_0}{s_0}(1 - p\frac{b}{b_0(1 + \frac{b}{b_0})})\\
% =&\frac{b_0}{s_0}(1 - p\frac{b}{b_0 + b})\\
% \end{align*}

Because \stasset is fungible, \emph{every} stakeholder is negatively affected,
proportionally to their holdings.
As everyone else was
wise, this constitutes unfair punishment.

The above attack requires the adversary to expend capital $b$ to cause
%capital bp ?
harm to others, and is irrational. In the remainder of this section, we
will explore how to make this attack profitable.

% TODO: conjecture that in the adversarial setting, the three properties
% (fungibility, proportional representation, fair punishment) are
% not all attainable when the protocol is slashably safe.

\noindent
\textbf{Making the attack profitable.}
The profitable version of the attack works similarly to the above
irrational attack, but with some extra steps. As before, the adversary, initially
holding a capital of $b$, begins by spawning the colluding validator $\mathcal{V}$,
delegates $b$ at time $t_2$, and equivocates at time $t_4$.

During equivocation, the adversary does not want to be holding any
\stasset of her own, as the price of \stasset is about to drop.
Therefore, at time $t_3$ (where $t_2 < t_3 < t_4$) the adversary sells
the acquired $s$ \stasset for $b$ \asset in the market.
% TODO: account for discount when selling \stasset for \asset
% If a loan was used to acquire her initial capital,
% these \asset can be used to repay it.
The adversary has now managed to add $b$ \asset delegated to validator $\mathcal{V}$
in the protocol's delegation pool while not currently holding
any \stasset. The loss has been averted.
% TODO: move to resignalling section?
% Since the protocol has Proportional Representation, the new $s$
% \stasset holder can eventually resignal his delegation intent and the
% $b$ \asset delegated to $\mathcal{V}$ will be redelegated elsewhere.
% % Explain how redelegation works?
% However, during the unbonding period $\delta$, the redelegated \asset will
% still be bonded with $\mathcal{V}$ and therefore vulnerable to slashing
% if $\mathcal{V}$ equivocates.

% TODO: remove "flash loan" label?
% TODO: mark b, s above black line and below red/blue lines
\import{./}{profitable-attack-diagram.tex}

A small extra trick will allow her to profit.
Before forcing the price of \stasset to drop, at time $t_0 < t_4$
the adversary \emph{shorts}
\stasset: She takes a loan of $z$ \stassets and
sells\footnote{Instead of \emph{selling}, the adversary can \emph{redeem}, but
this may incur an unbonding delay, which can be rectified by taking a loan.
See Section~\ref{sec:stasset-price}.}
them for $b^* = z \frac{b_0}{s_0}$ \asset in the market.

Lastly, at time $t_5 > t_4$ after the price drop, the adversary closes her short position by repaying the
\stasset loan. The duration of the loan was $\Delta_z = t_5 - t_0$, so the
total loan amount to be repaid, including the principal and interest, is
% TODO(tzinas): define $f$ here
$((1 + \rstasset)^{\Delta_z} + \betastasset) z$ \stasset.
To recover this amount of \stasset, the adversary \emph{deposits}
$b' = \frac{b_0}{s_0}(1 - p\frac{b}{b_0 + b}) ((1 + \rstasset)^{\Delta_z} + \betastasset) z$ \asset
into the protocol, which allows her to issue the exact required \stasset
to be paid back. This concludes the attack.

Her total profits from the attack are
$b^* - b' = \frac{b_0}{s_0}z(1 - (1 - p\frac{b}{b_0 + b})((1 + \rstasset)^{\Delta_z} + \betastasset))$.
This will be profitable if $(1 - p\frac{b}{b_0 + b})((1 + \rstasset)^{\Delta_z} + \betastasset) < 1$,
i.e., the cost $(1 + \rstasset)^{\Delta_z} + \betastasset$ of money borrowing (which is always $> 1$) is
sufficiently cheap that she can make up for it by the price movement
$1 - p\frac{b}{b_0 + b}$ she has caused.

% Maybe this should be moved to the appendix
\noindent
\textbf{Realizing profits in USD.}
If the attacker wants to do bookkeeping in a more stable reference currency,
such as USD, the attack is still profitable. The attacker begins by buying
\asset for USD. At the end of the attack, the attacker sells \asset for USD.
Because the attack concerns a particular liquid staking protocol, and
not the whole \asset network, the price of \asset will likely not
be significantly affected by the attack at all.
This attack decreases the market confidence in \stasset,
but not in \asset.
In fact, because
the attack causes slashing of \asset, the supply of \asset is decreased
and the price of \asset with respect to the reference currency may
even increase.
Lastly, any price fluctuations of \asset with respect USD will likely be
minor, as the attack has a short duration of a couple of seconds.

% TODO: resignalling

% TODO: consider the case p < 1

\subsection{The Market price of \stasset}\label{sec:stasset-price}
% TODO: Consider moving this to appendix

Let us consider the price $\alpha$ of \stasset denominated in \asset in the market.
Because the option always exists to mint at a rate of $\frac{s_0}{b_0}$ by
depositing, the price of \stasset denominated in \asset in a perfectly
efficient market is $\frac{b_0}{s_0}$ at maximum. Otherwise, no
rational buyer would use the market. Hence, the market rate is
$\alpha \leq \frac{b_0}{s_0}$.

There are two options to convert $s$ \stasset to \asset: either sell
at the market rate to obtain $b = \alpha s$, or use the redemption mechanism.
Using the redemption mechanism, the \assets become available after
time $\delta$.
Initially, using $s$ \stasset, a redemption is made of $b' = s \frac{b_0}{s_0}$
delegated assets. To get $b$ \asset immediately (and avoid having to wait
for the unbonding period), a loan of $b$ \asset is taken~\cite{liquid-staking-report} and
repayed after duration $\delta$. The amount of \asset that needs to be paid back,
including principal and interest, is $((1 + \rasset)^\delta + \betaasset)b$ \asset.
We set this amount to be equal to $b'$, the amount of \assets that will be
unbonded after $\delta$ time. Solving for $b$, we get
$b = s \frac{b_0}{s_0 ((1 + \rasset)^\delta + \betaasset)}$.
Therefore, in an efficient market $\alpha \geq \frac{b_0}{s_0 ((1 + \rasset)^\delta + \betaasset)}$.
We deduce that the bounds for an efficient market of \asset and \stasset are

\[
  \frac{b_0}{s_0 ((1 + \rasset)^\delta + \betaasset)} \leq \alpha \leq \frac{b_0}{s_0}\,.
\]

Some liquid staking protocols use the principle of remittances to match depositing and
redeeming parties, so that the redeeming party does not have to incur any unbonding delay.
If the redeemed amounts exceed the deposited amounts, some amounts will necessarily uncur
a delay. However, the above bounds may effectively be tighter than $\delta$ due to
a shorter effective unbonding duration.

% TODO: explore whether the principle of remittances is actually implemented anywhere

% TODO: mention in a remark (maybe later) that exchange rates between asset/stAsset may not
% be 1:1 because slashing risk and unbonding time must be factored in, and cite
% the economist's paper on liquid staking:
% https://papers.ssrn.com/sol3/papers.cfm?abstract_id=4180341

% TODO: Study a few of the big Liquid Staking protocols (https://defillama.com/protocols/Liquid%20Staking)
% top 13 e.g., Marinade, Algo Liquid, Benqi, Stader, Ankr

% TODO: Re-state the attack formulae using the corrected pricing of stasset

% Instead of selling the \stasset in the open market, maybe the
% adversary could use the
% protocol's redemption mechanism and receive delegated \asset
% instead of \asset
% Delegated \asset can also be used to
% continue with the attack if the protocol accepts already delegated
% \asset in addition to \asset when liquid staking.

% TODO: optimization problem, find optimal balance between b/c/z
% given limited capital.
