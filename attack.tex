\subsection{Attack}
% TODO: conjecture that in the adversarial setting, the three properties
% (fungibility, proportional representation, fair punishment) are
% not all attainable when the protocol is slashably safe.

We now describe an attack an adversary can conduct which leverages the
Principal--Agent problem of liquid staking.

\noindent
\textbf{Attack overview.}
The core idea is that the adversary can create her own validator
which she will collude with.
Even though no one else will choose to delegate to
this validator, the adversary can delegate to this validator herself.
The adversary begins the attack with some initial capital of \asset, which
can be her own capital or borrowed capital. The goal is to increase her
\asset holdings. She liquid stakes her capital, signaling delegation
intent to her own validator and receiving \stasset in return.
She sells these tokens in the open market in return for \asset.
Next, she shorts \stasset and equivocates using her own validator.
This causes a slashing of parts of the \asset holdings of the liquid
staking protocol, and therefore a drop in the price of \stasset
as denominated in \asset. Lastly, the adversary closes her short
position and realizes her profits.

% \begin{theorem}
% Any fungible liquid staking protocol with Proportional Representation
% deployed over any proof-of-stake consensus protocol with slashable safety
% cannot have fair punishment.
% \end{theorem}

In more detail, the attack works as follows.
Let $b_0$ be the amount of delegated \asset in the protocol's delegation pool,
and $s_0$ be the total amount of \stasset tokens outstanding
before the attack commences.
Initially,
the adversary $\mathcal{A}$ creates a new validator $\mathcal{V}$
under her control\footnote{To create this validator, she uses a fresh identity
to suppress potential suspicions. While most validators
have a real-world presence and can be held legally
accountable~\cite{liquid-staking-report}, this validator is pseudonymous.}.
We do not require any of the existing protocol participants to delegate
to this validator for the attack to work, i.e., we assume all participants
are wise and all other validators are honest.

At the beginning of the attack time $t_0$, the adversary obtains capital
$b$ \asset. Then, $\mathcal{A}$
liquid stakes $b$ \asset to the protocol
signaling delegation intent to $\mathcal{V}$. Because the
protocol has Proportional Representation, it delegates $b$ \asset to $\mathcal{V}$.
The protocol now holds $b$ delegated \asset.
In return, $\mathcal{A}$ receives $s = b \frac{s_0}{b_0}$ \stasset.
At this moment, the quoted protocol price of $1$ \stasset is
$\frac{b_0}{s_0}$ \asset.

Next, the adversary sells $s$ \stasset for $b$ \asset in the market.
% TODO: account for discount when selling \stasset for \asset
% If a loan was used to acquire her initial capital,
% these \asset can be used to repay it.
Up to this point, the adversary
has managed to add in the protocol's delegation pool $b$
\asset delegated to validator $\mathcal{V}$ while not currently holding
any \stasset. The quoted protocol price of $1$ \stasset after this process
is $\frac{b_0 + b}{s_0 + s}$ \asset.

Before forcing the price of \stasset to drop, the adversary \emph{shorts}
\stasset: She takes a loan of $z$ \stasset and sells them for
$z \frac{b_0 + b}{s_0 + s}$ \asset in the market.
% TODO: check that this price is achievable or if there is a discount
Then, the adversary equivocates using $\mathcal{V}$. This causes the
slashing of $qb$ \asset from the protocol's delegation pool. The new
amount of delegated \asset in the pool drops to $b_0 + (1 - q) b$, however
the amount of \stasset circulating in the market remains the same. Hence,
the new price of $1$ \stasset is $\frac{b_0 + (1 - q) b}{s_0 + s}$ \asset.
At this point, the adversary deposits part of her \asset in the protocol
in order to get $z$ \stasset and close her short position by repaying the loan.
The rest of the \asset holdings are her profit.

% Instead of selling the \stasset in the open market, maybe the
% adversary could use the
% protocol's redemption mechanism and receive delegated \asset
% instead of \asset
% Delegated \asset can also be used to
% continue with the attack if the protocol accepts already delegated
% \asset in addition to \asset when liquid staking.

\import{./}{attack-diagram.tex}

\noindent
\textbf{Realizing profits in USD.}
If the attacker wants to do bookkeeping in a more stable reference currency,
such as USD, the attack is still profitable. The attacker begins by buying
\asset for USD. At the end of the attack, the attacker sells \asset for USD.
Because the attack concerns a particular liquid staking protocol, and
not the whole \asset network, the price of \asset will likely not
be significantly affected by the attack at all.
This attack decreases the market confidence in \stasset,
but not in \asset.
In fact, because
the attack causes slashing of \asset, the supply of \asset is decreased
and the price of \asset with respect to the reference currency may
even increase.
Lastly, any price fluctuations of \asset with respect USD will likely be
minor, as the attack has a short duration of a couple of seconds.
