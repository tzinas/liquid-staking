% TODO (w/dionyziz): clean up
\section{Liquidity Staking Module}\label{sec:lsm}

% TODO (tzinas): make this into a figure
% Stake -> Delegated Stake -> Tokenized Stake -> Refungibilized Stake
% .    \
% .     -> Exempt Delegation

In the context
of Cosmos, the exempt delegation mechanism is planned to be applied at the consensus
layer by the Liquidity Staking Module (LSM)~\cite{liquidity-staking-module}.
When this mechanism is used, \emph{assets} are first delegated to a validator by
a principal who obtains \emph{delegated assets}, marking them as exempt or non-exempt.
In the case of non-exempt delegated assets, these are then
\emph{tokenized} into \emph{LSM shares}, representations of delegated assets that
are minimally fungible (fungible among the other tokens that were delegated in the
same batch to the same validator). These tokenized
shares are subject to the exempt delegation constraint $b \leq \phi c$. The tokenized shares
can then be \emph{deposited} into the liquid staking protocol, which issues liquid staking
tokens (\stassets), as usual, in a process termed \emph{refungibilization}.
The protocol does not need to delegate further, as it can readily start reaping the
delegation rewards (as long as a relevant so-called \emph{LSM record} is also transferred
along with the tokenized shares).
It also does not need to perform further exempt delegation constraint checks.
When the user \emph{redeems} \stassets, the protocol may elect to give
back tokenized shares instead of \assets. Those can then be unwrapped into delegated assets,
that can afterwards be undelegated into \assets after the relevant unbonding
period expires.
Through this mechanism, the exempt delegation $c$ of a validator is
a \emph{shared} amount across potentially multiple liquid staking
protocols that opt to accept tokenized shares
instead of \assets directly. The \emph{intent} necessary for proportional
representation can be read by the liquid staking protocol by simply
looking at the LSM tokenized share records, and no separate voting is
necessary when entering the protocol. The factor $\phi$ is decided not by the
liquid staking protocols' governance, but by the governance of
the underlying chain. The slashing factor $q$ is applied directly
by the chain and not by the liquid staking protocol.
In the current LSM design, $q = p$.
If a liquid staking protocol participates in
multiple chains, the $\phi$ factors can be different in each chain.
In our exposition, we abstract out these implementation details to highlight
the economic issues at hand.

% First, the stakeholder delegates their stake
% directly to the validator of their choice. They are now holders of \emph{delegated
% stake}, which is non-fungible and illiquid. They then
% transfer\footnote{In some Cosmos implementations, illiquid delegated stake requires an intermediate
% step before it can transferred: making the delegated stake \emph{minimally fungible}.
% This is done by first converting the delegated stake into minimally fungible
% \emph{liquid staking tokens} using the Liquidity Staking Module (LSM)~\cite{liquidity-staking-module}.
% These tokens are then transfered to the liquid staking protocol for refungibilization.}
% this delegated stake to the liquid staking protocol, which issues derivative
% tokens to the stakeholder. These tokens are fully fungible. We call this
% process \emph{refungibilization}, because the user starts with fungible assets,
% which we denote \textsf{Asset}, converts them to non-fungible delegated stake,
% which we denote \textsf{dAsset}, and, finally, deposites these into the liquid
% staking protocol to receive a fully fungible derivative token, which we denote
% \textsf{stAsset}.
