\pdfbookmark[subsection]{Liquid Staking Module (LSM)}{lsm}
\section{Liquid Staking Module (LSM)}\label{sec:lsm}
In the context
of Cosmos, the exempt delegation mechanism is planned to be applied at the consensus
layer by the Liquidity Staking Module (LSM)~\cite{liquidity-staking-module}.
When this mechanism is used, \emph{assets} are first delegated to a validator by
a principal who obtains \emph{delegated assets}, marking them as exempt or non-exempt.
In the case of non-exempt delegated assets, these are then
\emph{tokenized} into \emph{LSM shares}, representations of delegated assets that
are minimally fungible (fungible among the other tokens that were delegated in the
same batch to the same validator). These tokenized
shares are subject to the exempt delegation constraint $b \leq \phi c$. The tokenized shares
can then be \emph{deposited} into the liquid staking protocol, which issues liquid staking
tokens (\stassets), as usual, in a process termed \emph{refungibilization}.
The protocol does not need to delegate further, as it can readily start reaping the
delegation rewards (as long as a relevant so-called \emph{LSM record} is also transferred
along with the tokenized shares).
It also does not need to perform further exempt delegation constraint checks.
When the user \emph{redeems} \stassets, the protocol may elect to give
back tokenized shares instead of \assets. Those can then be unwrapped into delegated assets,
that can afterwards be undelegated into \assets after the relevant unbonding
period expires.
Through this mechanism, the exempt delegation $c$ of a validator is
a \emph{shared} amount across potentially multiple liquid staking
protocols that opt to accept tokenized shares
instead of \assets directly. The \emph{intent} necessary for proportional
representation can be read by the liquid staking protocol by simply
looking at the LSM tokenized share records, and no separate voting is
necessary when entering the protocol. The factor $\phi$ is decided not by the
liquid staking protocols' governance, but by the governance of
the underlying chain. The slashing factor $q$ is applied directly
by the chain and not by the liquid staking protocol.
In the current\footnote{Zaki Manian, personal communication, Jan 1st, 2023}
LSM design, $q = p$.
% TODO (w/dionyziz): Is this citation correct?
If a liquid staking protocol participates in
multiple chains, the $\phi$ factors can be different in each chain.
In our exposition, we abstract out these implementation details to highlight
the economic issues at hand.