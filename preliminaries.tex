\section{Preliminaries}\label{sec:preliminaries}

% TODO: citations

\noindent
\textbf{Proof-of-Stake.} In Proof-of-Stake (PoS) systems, ledgers are secured
by \emph{validators} who participate in the protocol by proposing and voting
on blocks. To become a validator, a stakeholder must \emph{bond} their stake
which locks it up for a particular period of time. Validators promise they will
not \emph{equivocate} by signing conflicting blocks.
In case of equivocation, a percentage $0 < p \leq 1$ of the locked stake is
\emph{slashed} and the validator is permanently deactivated. In the cryptographic
model, validators can be \emph{honest} or \emph{adversarial}. The honest validators
run the prescribed protocol, and hence never equivocate, whereas the adversarial
validators can deviate from the protocol arbitrarily.

% TODO: accountable/slashable safety

\noindent
\textbf{Delegation.} Since not everyone has the capacity to become a validator,
a stakeholder can \emph{delegate} their stake to a validator to participate in
the staking process in their stead. The voting power of the validator accounts
for the delegated stake, and delegated stake is also slashed in case of validator
misbehavior. The stake bonded by a validator themselves and not delegated from
others is known as \emph{self-delegation}. Self-delegations as
well as stake delegated from others is known as \emph{delegated stake},
and the capital holder of delegated stake is known as the \emph{delegator}.
A delegator can \emph{undelegate} at any time, but must wait for an
\emph{unbonding period}.

\noindent
\textbf{Liquid Staking.} Delegated stake earns rewards, but remains locked and
is illiquid. Delegators often wish to use their delegated stake as collateral,
for example to take loans~\cite{gudgeon2020defi} or to, more broadly, participate in the
DeFi~\cite{defi-sok} economy. Protocols that enable this ability are known as
\emph{liquid staking protocols}~\cite{liquid-staking-report}.
Some such protocols~\cite{lido,stride} operate
in the form of smart contracts (in Ethereum) or separate appchains (in Cosmos).
Stakeholders deposit their funds into the liquid staking \emph{pool} maintained
by the liquid staking protocol. Upon deposit, these contracts act as
delegators and delegate the incoming funds to their choice of validators, holding
onto the delegated stake. During the deposit, a new derivative asset is minted,
which is given to the depositor as a claim to the delegated stake held by the
liquid staking contract. Such derivative tokens, when issued from the same liquid
staking contract, are fungible with one another (\eg \textsf{stATOM} in Stride).

% TODO: how does the governance process for validator choice work in Stride, Lido, Persistence?

The choice of which validator the liquid staking protocol's assets are delegated to
depends on the particularities of the protocol, but, broadly speaking
is either centralized, or decided through a voting process among a governance board.
Typically, this governance board consists of governors who are determined by the holdings
of a governance token. This governance token is rewarded to liquid stake holders and accumulated
on a continuous basis proportionally to the size of their deposits into the pool.
In the voting process, anyone can propose for a proportion of the pool's assets
to be delegated to a validator of their choosing.
Each governor can then vote \emph{yes} or \emph{no}
to a proposal, and decisions are taken by weighted majority
Details of the governance process
that vary per protocol include the threshold to achieve majority, and the threshold to
veto. It is possible that a particular governor is a past, but not current, protocol user.

\noindent
\textbf{Loans.} For the attack we will describe in this paper, some upfront
capital is required. A portion of this capital is needed only throughout \emph{one}
transaction. Towards this purpose, a \emph{flash loan}~\cite{gudgeon2020defi} can
be obtained, in which the loan provided does not risk any funds. We assume that
a loan of duration $\Delta$ for capital $c$ has an \emph{upfront} cost $\beta$
and an \emph{interest rate} $r$ for a total cost of
$(1 + r)^\Delta c - 1 + \beta$ (the term $\beta$ models the cost of a flash loan,
which has duration $\Delta = 0$).

\begin{itemize}
  \item LSM (tokens)
  \item Quicksilver (refungibilization)
  \item Exempt Delegations
\end{itemize}
