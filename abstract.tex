\begin{abstract}
  Proof-of-Stake systems require stakers to lock up their funds in
  order to participate in the consensus validation process. This
  leads to capital inefficiency, as locked capital cannot be used for
  other DeFi applications. Liquid staking rewards stakers with fungible
  tokens in return for locking up their assets, which can in turn be
  reused in the DeFi economy. However, liquid staking introduces
  unexpected risks, as all delegated stake is now fungible. This
  exacerbates the already existing Agent--Principal problem which
  is faced during delegation. In this short paper, we put forth
  a framework of four desired properties for liquid staking systems:
  Decentralization, Fungibility, Fairness, and Security.
  We analyze existing liquid staking platforms in this framework.
  We highlight the
  risks introduced by Liquid Staking and study a particular solution
  proposed by the Cosmos ecosystem: Exempt delegations. Lastly, we give concrete
  evaluations of how exempt delegations must be calculated in order to
  mitigate risk.
\end{abstract}
