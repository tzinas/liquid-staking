\begin{abstract}
  Proof-of-Stake systems require stakers to lock up their funds in
  order to participate in consensus validation. This
  leads to capital inefficiency, as locked capital cannot be invested in
  Decentralized Finance (DeFi).
  \emph{Liquid staking} rewards stakers with fungible
  tokens in return for locking up their assets, which can in turn be
  reused in the DeFi economy. However, liquid staking introduces
  unexpected risks, as all delegated stake is now fungible. This
  exacerbates the already existing Principal--Agent problem
  faced during any delegation. In this short paper, we
  study the Principal--Agent problem in
  the context of liquid staking, and explain why many liquid staking
  protocols choose to use a centralized or federated governance
  mechanism. We highlight the dilemma
  between the choice of \emph{proportional representation}
  (having one's stake delegated to the validator of choice)
  and \emph{fair punishment}
  (being economically affected only when one's choice is misinformed).
  We put forth an attack illustrating that these two notions are
  fundamentally incompatible in an adversarial setting.
  We then describe the mechanism of exempt delegations,
  used by some staking systems today, and devise a precise
  formula for quantifying the correct choice of exempt delegation
  which allows balancing the two notions of fairness
  in the rational model.
\end{abstract}
